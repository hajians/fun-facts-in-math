\documentclass[11pt]{article}

\usepackage{sectsty} \usepackage{graphicx} \usepackage{ulem}
\usepackage{amsmath} \usepackage{amsfonts}

\newcommand{\R}{\mathbb{R}} \newcommand{\Ex}{\mathbb{E}}
\newcommand{\Pro}{\mathbb{P}} \newcommand{\D}{\mathcal{D}}
\newcommand{\q}{\textsf{q}} \newcommand{\Lsp}{\mathrm{L}}
\newcommand{\dx}{\text{d}x} \newcommand{\norm}[1]{\Vert #1 \Vert}

% Margins
%% \topmargin=-0.45in
%% \evensidemargin=0in
%% \oddsidemargin=0in
%% \textwidth=6.5in
%% \textheight=9.0in
%% \headsep=0.25in

\title{Fun facts in mathematics} \author{Soheil Hajian} \date{\today}

\begin{document}
\maketitle
% Optional TOC
\tableofcontents
\pagebreak

%--Paper--
\section{(Numerical) linear algebra}
\subsection{Rayleigh quotient} \label{sec:rayleigh}
\section{Probability theory}
\subsection{Chebyshev inequality}
Let $X$ be a continuous random variable with density function
$f_X(x)$, mean $\mu$ and standard deviation $\sigma$. Then the
following inequality holds
\begin{equation} \label{eq:chebyshev-ineq}
  \Pro(|X-\mu| \geq k \sigma) \leq \frac{1}{k^2}, \quad \forall k \geq
  1.
\end{equation}
In order to prove (\ref{eq:chebyshev-ineq}) we use definition of the
probability and standard deviation of $X$:
\begin{equation}
  \begin{array}{rcl}
    \Pro(|X-\mu| \geq k \sigma) &=& \int_{|x-\mu|\geq k\sigma} f_X(x)
    \dx \\ &\leq& \int_{|x-\mu|\geq k\sigma}
    \frac{|x-\mu|^2}{(k\sigma)^2} f_X(x) \dx \\ &\leq& \int_{\R}
    \frac{|x-\mu|^2}{(k\sigma)^2} f_X(x) \dx \\ &=&
    \frac{\sigma^2}{(k\sigma)^2} \\ &=& \frac{1}{k^2}.
  \end{array}
\end{equation}
%
\subsection{Law of large numbers}
Let us consider a sequence of independent and identically distributed
(i.i.d.) samples $(X_1, X_2, \dots, X_n)$ from a common random
variable $X$. The law of large numbers states that the mean of the
above sequence converges to the mean of $X$ as $n$ grows to
infinity. That is
\begin{equation}
  \bar{X}_n \rightarrow \mu \text{ as } n \rightarrow \infty,
\end{equation}
where
\begin{equation}
  \bar{X}_n = \frac{X_1 + X_2 + \cdots + X_n}{n},
\end{equation}
and $\mu = \mathbb{E}(X)$. Convergence should be understood in the
sense of almost surely (a.s.) which corresponds to the strong law of
large numbers. The weak law of large numbers corresponds to the
convergence in probability of the mean of the samples to the mean of
$X$.

Here we prove the weak law of large numbers for the case when $X$ is
continuous random variable. From Chebyshev inequality
(\ref{eq:chebyshev-ineq}) we have for the random variable $\bar{X}_n$
\begin{equation} 
  \Pro(|\bar{X}_n-\mu| \geq \varepsilon) \leq
  \frac{\sigma_{\bar{X}_n}^2}{\varepsilon^2},
\end{equation}
where we set $\varepsilon = k \sigma_{\bar{X}_n}$. On the other hand
from the definition of $\bar{X}_i$ we can conclude that
\begin{equation*}
  \sigma_{\bar{X}_n} = \frac{1}{\sqrt{n}} \sigma_X.
\end{equation*}
If $\sigma_X$ is finite we can conclude that
\begin{equation*} 
  \Pro(|\bar{X}_n-\mu| \geq \varepsilon) \leq \frac{\sigma_{X}^2}{n
    \varepsilon^2},
\end{equation*}
and therefore
\begin{equation} 
  1- \frac{\sigma_{X}^2}{n \varepsilon^2} \leq \Pro(|\bar{X}_n-\mu|
  \leq \varepsilon) \leq 1, \quad \forall \varepsilon > 0.
\end{equation}
Letting $n \rightarrow 0$ yields that $ \Pro(|\bar{X}_n-\mu| \leq
\varepsilon) \rightarrow 1$ for all $\varepsilon > 0$.

\section{Pattern recognition}
\subsection{Linear discriminant analysis}
In this section we introduce a method to find a subspace that that
aims to maximize the distance between data points belonging to
different classes (intra-classes distance) while minimizing the
distance of data points belonging to the same class (inter-classes).

Let us denote the dataset by the tuple $(X, y)$ where $X \in \R^{n
  \times p}$ is the so-called feature matrix, and $y \in \R^{n}$ is
the target vector. Here $p$ denotes the number of data points in the
dataset and $n$ is the dimension of the feature space.

Each column of $X$ corresponds to the features of a data point which
we denote by $x^{(j)} \in \R^{n}$, and similarly each entry of $y$,
i.e., $y^{(j)}$ corresponds to the class of the data point $j$, for
all $j=1,\dots, p$. We consider that each data point can belong to one
and only one class, and we denote the total number of classes by
$K$. That is $y^{(j)} \in \{1,\dots, K\}$ for all $j=1,\dots, p$.


\subsubsection{Scatter matrices and spread}
Let us denote the projection of the features of a data point,
$x^{(j)}$, into a normalized vector $\q \in \R^{n}$ by $z^{(j)}$, and
for all data points by $z := \q^\top X$. A measure of the spread of
projected values can be define by the Euclidean $2$-norm,
$\norm{z}_2$:
\begin{equation}
  \norm{z}_2 = \q^\top X X^\top \q.
\end{equation}
The matrix $X^\top X$ is called the scatter matrix and we denote it by
$S \in \R^{n\times n}$. Note that from the definition of $S$ we can
conclude that $S$ is symmetric and at least positive
semi-definite. Therefore maximizing the spread corresponds to finding
the normalized vector $\q$ that maximizes $\q^\top S \q$, which by
Rayleigh quotient (see Section \ref{sec:rayleigh}) corresponds to the
eigenvector of $S$ corresponding to the maximum eigenvalue of $S$.

We will now define two metrics: intra-class and inter-class spread of
the dataset. It is convenient to define first the set of data points
that belong to the same class. Let us define the set of data points
belonging to the same class by
\begin{equation}
  \D_k := \big\{ j : y^{(j)} = k, \quad \forall j=1,\dots,p \big\},
\end{equation}
for all $k = 1, \dots, K$. This then motivates to rearrange the
feature matrix, $X$, such that data points belonging to the same class
be adjacent to each other:
\begin{equation}
  X = [X_1, X_2, \dots, X_K],
\end{equation}
and similarly the target vector $y = (y_1, \dots, y_K)^\top$. Note
that each matrix $X_k$ can have a different size, i.e., $X_k \in \R^{n
  \times p_k}$ for all $k = 1, \dots, K$.

In order to define the intra-class spread, we first translate each
class the data points to the origin. The center of a class is defined
by
\begin{equation}
  c_k := \frac{1}{|\D_k|} \sum_{x \in \D_k} x.
\end{equation}
We then define the centered feature matrix of each class by
\begin{equation}
  X_{k, c} := \Big[x^{(j_1)}-c_k, \quad x^{(j_2)}-c_k, \quad \dots,
    \quad x^{(j_{p_k})}-c_k \Big],
\end{equation}
and the centered feature matrix by $X_w = [X_{1, c}, \dots, X_{K,
    c}]$.  The intra-class spread matrix is then defined by $S_w :=
X_w^{} X_w^{\top}$. We now focus on defining inter-class spread. Let
us define the global centroid of the data by
\begin{equation}
	c := \frac{1}{p} \sum_{i=1}^{p} x^{(i)}.
\end{equation}
We would like to measure the distance of the centroid of each cluster
to the global centroid. Therefore we define the following matrix
\begin{equation}
	\bar{X} := [ \underbrace{(c_1 - c), \dots, (c_1 - c)}_{p_1
            \text{ times}}, \dots, \underbrace{(c_k - c), \dots, (c_k
            - c)}_{p_k \text{ times}}] \in \R^{n \times p},
\end{equation}
whose columns measures the distance of each centroid to the global
centroid. Finally the inter-class spread matrix can be defined by
\begin{equation}
	S_b := \bar{X} \bar{X}^{\top}.
\end{equation}

As we discussed earlier in this section, we would like to maximize
inter-class spread while minimizing the intra-class spread. To do so
we can scalarize the above two metrics by defining the following
scalar function:
\begin{equation}
	H(\q) := \frac{ \q^{\top}_{} S_{b}^{} \q }{\q^{\top}_{}
          S_{w}^{} \q}, \quad \forall q \in \R^{n}, q \not = 0.
\end{equation}
Any direction $\q$ that maximizes $H(\q)$ can be viewed as a desirable
direction maximizes the ratio of the spread of inter-class over the
spread of the intra-class. Note that such direction does not
necessarily maximizes spread of inter-class and minimizes spread of
intra-class simultaneously.

Let us suppose for the moment that $S_w$ is symmetric positive
definite (s.p.d.). Then using Rayleigh quotient of Section
\ref{sec:rayleigh} we can conclude that the maximum value of $H(q)$
corresponds to the maximum eigenvalue of $S_w^{-1} S_b$ and the
optimum direction is the corresponding eigenvector of $S_w^{-1} S_b$,
i.e.,
\begin{equation}
	v_{\max} = \arg \max_{\q\in \R^{n}} H(\q),
\end{equation}
where
\begin{equation}
	S_w^{-1} S_b \, v_{\max} = \lambda_{\max} \, v_{\max}.
\end{equation}
For the case where $S_w$ is only semi-definite, we can use the
argument that $S_w^{-1}$ exists. One approach would be to regularize
$S_w$ to an s.p.d. matrix by adding an s.p.d. matrix, e.g.,
\begin{equation}
	S_{w,\varepsilon} := S_w + \varepsilon I, \quad \varepsilon >
        0.
\end{equation}
Here $\varepsilon$ is a small positive real number.
\end{document}
